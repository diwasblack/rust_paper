\documentclass[12pt, titlepage]{article}
\usepackage{cite}
\usepackage{hyperref}
\usepackage{listings}
\usepackage{setspace}

\title{\large{Research paper} on \\ \textbf{Rust}}
\author{Diwas Sharma}
\date{November 16, 2017}

% Set line spacing to 1.5
\onehalfspacing

\usepackage[margin=1in]{geometry}

\begin{document}

\maketitle

\section{History}
Rust started as a side project of Mozilla employee Graydon Hoarse in 2006. Mozilla started supporting the language in
2009 and announced it at Mozilla Annual Summit, July 2010\cite{ProjectServo}. The initial compiler for rust was written
in OCaml. And somewhere around 2011 the compiler for rust was ported to rustc which was itself written using rust.

The first stable release of rust occured in May 12, 2015\cite{AnnounceRust}. Rust is primarily supported
by Mozilla but the team of core developers for Rust consist of non-Mozilla members as well. The
code for rust language is open sourced and is being actively developed at \href{https://github.com/rust-lang/rust}{https://github.com/rust-lang/rust}.

\section{Programming Categories}
Rust is a multi paradigms language which supports functional and imperative programming techniques but not every concepts in object oriented languages
can be implementation with exact same properties in rust.

\subsection{Imperative language features in Rust}
The rust language has support for variables, loops, assignments, and program states which are similar to C/C++. The variables in rust are immutable by
default, so the programmer needs to explictily declare the mutable variable by using keyword \textit{mut}.

\begin{lstlisting}
fn main() {
    // x is immutable
    let x = 5;
    // y is mutable
    let mut y = 6;
    println!("The value of x is: {} and y is: {}", x, y);
}
\end{lstlisting}

\subsection{Functional language features in Rust}
Rust has support for iterators, enums, and pattern matching which are features commonly associated with
functional programming languages.

\begin{lstlisting}
fn main(){
    let values = vec![1, 2, 3];
    // Using iterator to loop through the values
    for value in values.iter() {
        println!("Value: {}", value);
    }
}
\end{lstlisting}

Apart from these, rust supports higher order functions and closures.
\begin{lstlisting}
fn main() {
    // Declaration of a closure
    let square_number = |x| x * x;
    let result = square_number(3);
    println!("Result: {}", result);
}
\end{lstlisting}

\subsection{Object oriented language features in Rust}
Rust has \textit{structs} and \textit{enums} that have the data and \textit{impl} blocks provides a way to define
methods on those entities. When used together they are able to provide the same functionality as objects in object oriented programming.
But they are not objects in a strict sense. Every module, types, functions, and members are
private by default in rust. The \textit{pub} keyword can be used to define which code blocks are to be public. Thus, using this
approach encapsulation can be implementation in a rust code. However, rust does not provide a way to define a struct that
inherits from another struct such that it contains the fields and method of the struct that it is trying to inherit. So,
rust does not support inheritance directly but there are solutions for cases when inheritance is desired.

\section{Grammer}
Rust's grammar is defined over Unicode codepoints, and is described by a dialect of Extended Backus-Naur Form (EBNF),
specifically a dialect of EBNF supported by common automated LL$(k)$ parsing tools such as llgen\cite{RustLang}.
The grammer for rust is pretty complex and is described at \href{https://doc.rust-lang.org/grammar.html}{https://doc.rust-lang.org/grammar.html}.e.g.

\subsection{Lexical Structure for Comments.}
\begin{lstlisting}
comment : block_comment | line_comment ;
block_comment : "/*" block_comment_body * "*/" ;
block_comment_body : [block_comment | character] * ;
line_comment : "//" non_eol * ;
\end{lstlisting}

\subsection{While loops}
\begin{lstlisting}
while_expr : [lifetime':']? "while" no_struct_literal_expr'{'block'}';
\end{lstlisting}

\subsection{For expression}
\begin{lstlisting}
for_expr : [lifetime':']? "for" pat "in" no_struct_literal_expr'{'block'}';
\end{lstlisting}

\section{Type System}
Rust is a statically typed language which is strong typed\cite{RustIntro} as well. But, it does allow a convenient way to
assign an initial value to a variable while declaring the type by offering type inference.

\begin{lstlisting}
// The type of x is inferred from the statement
let x = 10;
\end{lstlisting}

Rust does not provide coercion(implicit type conversion) between the primitive types. But, explict type conversion can
be done using \textit{as} keyword.

\begin{lstlisting}
fn main(){
    let x = 65.11_f32;
    // Cast floating point to 8 bit unsigned integer
    let y = x as u8;
    println!("x: {} and y: {}", x, y);
}
\end{lstlisting}

\section{Implementation}
The official implementation of rust is a compiled language which itself is written in rust. The rust
language uses Semantic Versioning 2.0\cite{SemVer} to label its release. And, the latest release of rust language is 1.21.0\cite{Rust1.21}.

\section{Memory Management}
Memory management in rust is done through a system of ownership with a set of rules that the
compiler checks at compile time. Ownership is the key feature in rust that enables it to
guarantee memory safety while avoiding garbage collector. And, it does so without adding any extra run time cost\cite{RustOwnership}.

\subsection{Ownership}
The ownership rules that the compiler uses at compile time are as follows:
\begin{enumerate}
    \item{Each value in Rust has a variable that’s called its owner.}
    \item{There can only be one owner at a time.}
    \item{When the owner goes out of scope, the value will be dropped.}
\end{enumerate}

Consider an example shown below.
\begin{lstlisting}
{
    // s is valid from line below
    let s = String::from("Hello world");
} // this scope is now over, and s is no longer valid
\end{lstlisting}

Variable s is the owner for the value "Hello world" which is invalidated when the scope ends. Once the owner
associated with a value is invalidated the value is dropped. Consider another example
given below
\begin{lstlisting}
{
    let s = String::from("Hello world");
    let s2 = s; // variable s is invalidated
}
\end{lstlisting}
For dynamic variables such as s, assignment operation invalidates the previous
variable so we could say that the ownership was moved from variable s to s2. Same logic is used when
passing dynamic variable to functions as parameters.

\bibliography{paper}
\bibliographystyle{plain}
\end{document}
